%% introduction.tex
%%

%% ==============================
\chapter{Introduction}
\label{ch:Introduction}
%% ==============================

%% ===========================
\section{Motivation}
\label{ch:Introduction:sec:Motivation}
%% ===========================

Word order is a general issue when we want to translate text from one language to the other. Different languages normally have different word orders and the difference could be tremendous, when two languages come from different language families, such as English and Chinese. Depend on the languages, different word orders could have very distinguish features. For example, $45\%$ of the languages in the world are \ac{SOV} languages. Unlike in English, verbs are put after object in these languages. Japanese is a popular language among them. Instead of saying \emph{The black cat climbed to the tree top}, people would say \emph{The black cat the tree top to climbed} in Japanese. Another example is Spanish, in which people often put the adjective after the modified nouns. Following example shows how people would order the words differently \citep{google}.

%\begin{table}[H]
%\centering
%\begin{tabular}{| r l |}
%\hline 
%English & The black cat climbed to the tree top. \Hstrut \Tstrut \\
%\Hstrut Japanese & The black cat the tree top to climbed. \\
%Spanish & The cat black climbed to the top tree. \Bstrut \\
%\hline
%\end{tabular}
%\caption{Word orders of three different languages}
%\end{table}

\begin{center}
\begin{tabular}{ r l }
English & The black cat climbed to the tree top. \\
Japanese & The black cat the tree top to climbed. \\
Spanish & The cat black climbed to the top tree. \\
\end{tabular}
\end{center}

Among all the languages, Chinese is one language which is very different from English, because they belong to different language families and have long period of separately development. Both languages have a \ac{SOV} order, but they also have a lot of differences in word order. Especially sentences in both languages can sometimes have completely different syntactic structures. The differences may involve long-distance or unstructured position changes.

Most state-of-the-art phrase-based \ac{SMT} systems use language model, phrase table or decoder to adjust the word order. Or an additional reordering model is used in the log-linear model for word reordering. However, these methods may have some disadvantages, such as some don't handle long-distance reordering, some don't handle unstructured reordering and some are rather time consuming.

Encouraged by the results from the paper \cite{short}, \cite{long} and \cite{tree}, we further propose a new data-driven, rule-based preordering method, which extracts and applies reordering rules based on syntax tree. The method is called \ac{MLT} reordering, which orders the constituents on multiple levels of the syntax tree all together. This preordering method rearranges the words in source language into a similar order as they are supposed to be in the target language before translation. With the appropriate word order, better translation quality can be achieved. Especially, our preordering method is very suitable for translation between language pairs like English and Chinese, which have very different word orders. Besides, the method can also be combined with the above mentioned rule-based reordering methods to achieve better translation quality. 

To be more accurate, Chinese is referred to Mandarin Chinese throughout this thesis, which is the official language in People's Republic of China and standardized by its government.

%% ===========================
\section{Objective and Contribution}
\label{ch:Introduction:sec:ObjectiveAndContribution}
%% ===========================

So the objective of this thesis can be defined as follows:

\begin{center}
\parbox[c]{0.8\textwidth}{
We establish a new data-driven rule-based preordering method for translation between English and Chinese, which is based on multi-level syntactic information. The method reorders the source text of a \ac{SMT} system before translation by using the information from word alignments, \ac{POS} tags and syntax trees of the training data. Also we evaluate this method by checking the resulted translation quality and by comparing it with some other rule-based preordering method.
}
\end{center}

The ground of this thesis are three papers about rule-based preordering: \cite{short}, \cite{long} and \cite{tree}. While their reordering methods are primarily designed and optimized for German or other languages with similar characteristics, they are not necessarily suitable for Chinese translation, which is a langauge that belongs to a completely different language family and has some very distinct features. Or at least, there may be still much space for improvement in Chinese translation.

In this context, we further explore the possibilities for a more suitable reordering method for Chinese. And we propose the \ac{MLT} reordering method, which extracts the rules by detecting position change of constituents on multiple levels of subtrees in syntax trees from parallel training data. And guided by the rules, we can reorder the new text by examining subtrees of the same structure.

We will also evaluate our reordering method and compare it with some other these methods. Through the evaluation and comparison, we'll have a thorough understanding of what our approach can achieve.
%?as well as other language pair


%Our method is primarily based on the syntax trees. We 

%The evaluation is based on the BLEU score of the translation. %?In addition, we'll also manually analyze some translation examples to show how our method is suitable for word reordering related to syntactic structure change, when it's used for translation between English and Chinese.

%conclusion goal is

%% ===========================
\section{Related Work}
\label{ch:Introduction:sec:RelatedWork}
%% ===========================

Word reordering is an important problem for statistical machine translation, which has long been addressed.

In a phrase-based \ac{SMT} system, there are several possibilities to change the word orders. Words can be reordered during the decoding phase by setting a window, which allows the decoder to choose the next word for translation. Reordering could also be influenced by the language model, because the language model give probability of how a certain word is likely to follow. Different language model may give different probability, which further influences the decision made by log-linear model. Other ways to change the word orders include using distance based reordering models or lexicalized reordering models \citep{tillmann2004, koehn2005}. The lexicalized reordering model reorders the phrases by using information of how the neighboring phrases change orientations.

The hierarchical phrase-based translation model \citep{hier} is especially suitable for Chinese translation, and provide very good translation results. It extracts hierarchical rules by using information of the syntactic structure. Phrases from different hierarchies, or so-called phrases of phrases, are reordered during the decoding. 

The idea of phrases on different hierarchies has inspired us to create this preordering method based on multiple levels of the syntax tree. Besides, we also hope to detach the reordering from decoding phase and do it separately in a pre-process before decoding, in order to reduce the time for translation. This kind of preordering approaches use linguistic information to modify the word orders. 

Reordering approaches can also be rule-based, which extracts different types of reordering rules by observing reordering patterns from the training data and apply the rules to the sentences to be translated. Depends on how the rules are defined, different information may be used such as word alignments, \ac{POS} tags, syntax trees, etc.

Some early approaches use manually defined reordering rules based on the linguistic information for particular languages \citep{collins2005clause, popovic2006pos,habash2007syntactic}. Especially \cite{syntactic} is based on Chinese and has analyzed the reordering cases between English and Chinese based on syntactic structure. Later come the data-driven methods \citep{zhang2007chunk, crego2008using}, which learn the reordering rules automatically. 

\cite{short} introduced the idea of extracting reordering rules from the POS tag sequences of training data and use them for reordering. \cite{long} went further, and developed a method for long-distance word reordering, which works good on German-English translation task due to the long-distance shift of verbs. The method extracts discontinuous reordering rules in addition to the continuous ones, which contains a placeholder to match several words and enables the word to shift cross long distance.

Afterwards, \cite{tree} introduced a novel approach to reorder the words based on syntax tree, which leads to further improvements on translation quality. The algorithm takes syntactic structure of the sentences into account and extract the rules from the syntax tree by detecting the reordering of child sequences. It also has the variant based only on part of the child sequences which is suitable for language with flat syntactic structure such as German.

Recently, \cite{google} introduced a novel classified preordering approach. Unlike existing preordering models, it trains feature-rich discriminative classifiers that directly predict the target side word order. It reorders the children in subtrees in a similar manner as the syntax tree based method, but trains different classifiers for nodes with different number of children.

However, these approaches which are bases on POS tag sequences or syntax trees are mostly designed for German and are not especially adapted for Chinese translation. As Chinese has very different word orders, a reordering approach, which can further explore the syntactic structure of Chinese and utilize this information for reordering, is desirable.

Oracle reordering has also shown values for evaluating the potential of preordering. \cite{metrics} introduced the permutation distance metrics which can be used to measure reordering quality. And \cite{birch2} described how we can construct permutations from the word alignment as oracle reordering.

%% ===========================
\section{Structure}
\label{ch:Introduction:sec:Structure}
%% ===========================

In this thesis, we first explain some fundamental concepts and knowledge in chapter~\ref{ch:Foundations}, which are relevant for understanding this thesis. Then we introduce our reordering methods in detail in chapter~\ref{ch:ReorderingApproach}, including the problems of translating between English and Chinese and the motivation of our reordering approach. The results and evaluation of our method are presented in chapter~\ref{ch:Evaluation}. In chapter~\ref{ch:Discussion} we conclude this work with an overall discussion of our approach and results. And we also point out the possible directions for future research.