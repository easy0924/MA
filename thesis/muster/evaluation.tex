%% evaluation.tex
%%

%% ==================
\chapter{Evaluation}
\label{ch:Evaluation}
%% ==================

We've conducted two sets of experiments to test our reordering methods. The first set of experiments is designed for testing the English-to-Chinese translation, which is described in the first section of this chapter. The second set of experiments is designed for the other translating direction, which is described in the second section. Through the experiments on both translating direction, we could get a better overview of our methods' effect.

Both sections are composed of three parts: experimental setup, result and evaluation. The first part describes details of the system configurations and experimental data. The second part shows the \hyperref[ch:Foundations:sec:bleu]{BLEU} scores of different systems for comparison. And the last part evaluates the improvement with translation examples from the experiments.


\section{English to Chinese Systems}
\label{ch:Evaluation:sec:enw}

\subsection{Experimental Setup}
We performed experiments with or without different reordering methods covering the English to Chines translation direction. The reordering methods included the reordering with short rules, long rules, tree rules and our MLT rules. The system was trained on news text from the LDC corpus and subtitles from TED talks. The development data and test data were both news text from the LDC corpus. The system was a phrase based SMT system, which used a 6-gram language model with Knersey-Ney smoothing. Besides the pre-reordering, no lexical reordering or other reordering method was used. The text was translated through a monotone decoder. The Chinese text were first segmented into words before use.
The reordering rules were extracted based on the alignment, POS tags and syntactic trees from the training data. One reference of the test data was used for evaluating the BLEU score. Table $4.1$ shows the data size in detail.

\begin{table}[H]
\centering
%\begin{tabular}{|l|r|r|r|}\hline
%Data Set & Sentence Count & Word Count & Data Size \\ \hline
%Training Data & 111K & \multicolumn{2}{c}{Multi-column} & \multicolumn{2}{c}{Multi-column}\\
% & & 1 & 2 & 3 & 4 \\ \hline
%\end{tabular}

\begin{tabular}{|ll|r|r|r|r|r|}
\hline
\multicolumn{2}{|l|}{\multirow{2}{*}{Data Set}} & \multirow{2}{*}{Sentence Count} & \multicolumn{2}{|c|}{Word Count} & \multicolumn{2}{|c|}{Size (Byte)}\\ \cline{4-7}
& & & English & Chinese & English & Chinese \\
\hline
\multirow{2}{*}{Training Data} & \multicolumn{1}{|l|}{LDC} & 303K & 10.96M & 8.56M & 60.88M & 47.27M \\ \cline{2-7}
& \multicolumn{1}{|l|}{TED} & 151K & 2.58M & 2.86M & 14.24M & 15.63K \\ \hline
\multicolumn{2}{|l|}{Development Data} & 919 & 30K & 25K & 164K & 142K \\ \hline
\multicolumn{2}{|l|}{Test Data} & 1663 & 47K & 38K & 263K & 220K \\ \hline

\end{tabular}

\caption{BLEU score overview of English to Chinese systems}
\end{table}

\subsection{Result}

\begin{table}[H]
\centering
\STautoround*{2}
\begin{spreadtab}{{tabular}{|l|r|r|}}\hline
@				& @BLEU Score & @Improvement \\ \hline
@Baseline		& 12.07 & \\ \hline
@+Short Rules	& 12.50 & :={b3 * 100 /b2 - 100} \% \\ \hline
@+Long Rules   & 12.99 & :={b4 * 100 /b2 - 100} \% \\ \hline
@+Tree Rules   & 13.38 & :={b5 * 100 /b2 - 100} \% \\ \hline
@+MLT Rules    & 13.81 & :={b6 * 100 /b2 - 100} \% \\ \hline
@Oracle Reordering & 18.58 & :={b7 * 100 /b2 - 100} \% \\ \hline
\hline
@Long Rules   & 12.31 & :={b8 * 100 /b2 - 100} \% \\ \hline
@Tree Rules   & 13.30 & :={b9 * 100 /b2 - 100} \% \\ \hline
@MLT Rules    & 13.68 & :={b10 * 100 /b2 - 100} \% \\ \hline
\end{spreadtab}
\caption{BLEU score overview of English to Chinese systems}
\end{table}

Table $4.2$ shows the BLEU scores for configurations with different reordering methods. The table consist of $2$ sections. the first row of the top section shows results of the baseline, which involves no reordering. In the following rows of the top section, different types of reordering rules are combined gradually, each type per row, and the improvements are showed. For example, the row with ``+MLT Rules'' presents the configuration with all the rule types including MLT rules and all the other rules in the rows above. All the improvements are calculated comparing to the baseline in percentage. Each row with a certain reordering type presents all the different variations of the type and the best score under these configurations are shown. For example, long rules also presents the left rules and right rules type. In the lower section of the table, rules types are not combined and the effect of each rule type is shown. %Besides, the lower section also shows the total number of rules that are extracted and applied for each rule type. (Problem: many variations :(  )
\subsection{Evaluation}

%% ===============================
\section{Experiement Setup}
\label{ch:Evaluation:sec:ExperimentSetup}
%% ===============================

\dots
Criterien
\cite{metrics}



%% ===============================
\section{English to Chinese Systems}
\label{ch:Evaluation:sec:zhen2}
%% ===============================


\subsection{Experimental Setup}

\subsection{Result}

\begin{table}[H]
\centering
\STautoround*{2}
\begin{spreadtab}{{tabular}{|l|r|r|}}\hline
@				& @BLEU Score & @Improvement \\ \hline
@Baseline		& 21.80 & \\ \hline
@+Short Rules	& 22.90 & :={b3 * 100 /b2 - 100} \% \\ \hline
@+Long Rules   & 23.13 & :={b4 * 100 /b2 - 100} \% \\ \hline
@+Tree Rules   & 23.84 & :={b5 * 100 /b2 - 100} \% \\ \hline
@+MLT Rules    & 24.14 & :={b6 * 100 /b2 - 100} \% \\ \hline
@Oracle Reordering & 26.80 & :={b7 * 100 /b2 - 100} \% \\ \hline
\hline
@Long Rules   & 22.10 & :={b4 * 100 /b2 - 100} \% \\ \hline
@Tree Rules   & 23.35 & :={b5 * 100 /b2 - 100} \% \\ \hline
@MLT Rules    & 24.14 & :={b6 * 100 /b2 - 100} \% \\ \hline
\end{spreadtab}
\caption{BLEU score overview of Chinese to English systems}
\end{table}

\subsection{Evaluation}

%
%%% ===============================
%\section{Effect dependency tree?}
%\label{ch:Evaluation:sec:e0}
%%% ===============================
%
%
%
%%% ===============================
%\section{Alternative Scanning?}
%\label{ch:Evaluation:sec:a}
%%% ===============================
%
%%% ===============================
%\section{Effect of Different Left Side}
%\label{ch:Evaluation:sec:e1}
%%% ===============================
%
%%% ===============================
%\section{Effect of Different Threshold}
%\label{ch:Evaluation:sec:e2}
%%% ===============================
%
%%% ===============================
%\section{Research on other language pair}
%\label{ch:Evaluation:sec:e3}
%%% ===============================

%
%%==============En to De systems================
%\begin{table}[H]
%\centering
%\STautoround*{2}
%\begin{spreadtab}{{tabular}{|l|r|r|}}\hline
%@				& @BLEU Score & @Improvement \\ \hline
%@Baseline		& 18.45 & \\ \hline
%@+Short Rules	& 19.09 & :={b3 * 100 /b2 - 100} \% \\ \hline
%@+Long Rules   & 19.16 & :={b4 * 100 /b2 - 100} \% \\ \hline
%@+Tree Rules   & 19.34 & :={b5 * 100 /b2 - 100} \% \\ \hline
%@+MLT Rules    & 20 & :={b6 * 100 /b2 - 100} \% \\ \hline
%@Oracle Reordering & 21 & :={b7 * 100 /b2 - 100} \% \\ \hline
%\end{spreadtab}
%\caption{Results of English to German translation}
%\end{table}
%
%\begin{table}[H]
%\centering
%\STautoround*{2}
%\begin{spreadtab}{{tabular}{|l|r|r|}}\hline
%@				& @BLEU Score & @Improvement \\ \hline
%@Baseline		& 18.45 & \\ \hline
%@+Short Rules	& 19.09 & :={b3 * 100 /b2 - 100} \% \\ \hline
%@+Long Rules   & 19.16& :={b4 * 100 /b2 - 100} \% \\ \hline
%@+Tree Rules   & 19.34 & :={b5 * 100 /b2 - 100} \% \\ \hline
%@+MLT Rules    & 1 & :={b6 * 100 /b2 - 100} \% \\ \hline
%@Oracle Reordering & 1 & :={b7 * 100 /b2 - 100} \% \\ \hline
%\end{spreadtab}
%\caption{Results of German to English translation}
%\end{table}
%%==============================


%%% ===============================
%\section{Experiement Result}
%\label{ch:Evaluation:sec:ExperimentResult}
%%% ===============================
%
%%% ===============================
%\section{Evaluation}
%\label{ch:Evaluation:sec:Evaluation}
%%% ===============================

\section{Conclusion}
\label{ch:Evaluation:sec:Conclusion}